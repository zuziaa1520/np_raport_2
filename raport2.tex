\documentclass[a4paper,onecolumn,twoside,10pt]{article}%{mwrep}

\usepackage{times}
\usepackage[utf8x]{inputenc}
\usepackage[T1]{fontenc}
\usepackage[polish]{babel}
\usepackage{lmodern} %Type1-font for non-english texts and characters
\usepackage{setspace}
\usepackage{enumitem}
\usepackage{subcaption}

\usepackage{booktabs} % Dla ładniejszych linii poziomych (\toprule, \midrule)
\usepackage{colortbl} % Dla kolorowania wierszy
\usepackage{array}    % Użyteczne z p{...}


\usepackage[cmex10]{amsmath}

%% Packages for Graphics & Figures %%%%%%%%%%%%%%%%%%%%%%%%%%

%\usepackage{bmpsize}

\usepackage{graphicx} %%For loading graphic files
\usepackage[pdf]{pstricks}
\usepackage{pst-all}
\usepackage{moredefs}
%\usepackage{auto-pst-pdf}
%\usepackage{auto-pst-pdf}
\usepackage[crop=off]{auto-pst-pdf}

\usepackage{fancyhdr}
\usepackage{url}
\usepackage{float}
\usepackage{color}
\usepackage{xcolor}

\usepackage{multirow}

%\usepackage{epstopdf}
\definecolor{light-gray}{gray}{0.4}
\definecolor{mauve}{rgb}{0.88, 0.69, 1.0}
\definecolor{pakistangreen}{rgb}{0.0, 0.4, 0.0}
\definecolor{pearl}{rgb}{0.94, 0.92, 0.84}
\definecolor{whitesmoke}{rgb}{0.96, 0.96, 0.96}
\definecolor{gray-pp}{rgb}{0.13, 0.6, 0.82}

\usepackage{colortbl}
\usepackage{listings}
\lstset{
	basicstyle=\footnotesize\ttfamily,
	columns=fullflexible,
	frame=single,
	breaklines=true,
	numbers=left, stepnumber=2, numbersep=5pt,
	numberstyle=\tiny\color{gray},
	keywordstyle=\color{blue},
	commentstyle=\color{gray-pp},
	stringstyle=\color{mauve},
	backgroundcolor = \color{whitesmoke},
	breakatwhitespace=true,
	showspaces=false,                % show spaces everywhere adding particular underscores; it overrides 'showstringspaces'
	showstringspaces=false,          % underline spaces within strings only
	showtabs=false,                  % show tabs within strings adding particular underscores
	postbreak=\mbox{\textcolor{red}{$\hookrightarrow$}\space}
}



\cfoot{-~\thepage \textcolor{light-gray}{~| Strona~-}}

\hyphenpenalty=10000		% nie dziel wyrazów zbyt często
\clubpenalty=10000			% kara za sierotki
\widowpenalty=10000			% nie pozostawiaj wdów
\brokenpenalty=10000		% nie dziel wyrazów między stronami
\exhyphenpenalty=999999		% nie dziel słów z myślnikiem
\righthyphenmin=3			% dziel minimum 3 litery

\tolerance=4500
\pretolerance=250
\hfuzz=1.5pt
\hbadness=1450

\sloppy						% umacnia pozycję prawego marginesu

\setlength{\textwidth}{\paperwidth}
\addtolength{\textwidth}{-5cm}
\setlength{\textheight}{\paperheight}
\addtolength{\textheight}{-5cm}
\setlength{\oddsidemargin}{0cm}
\setlength{\evensidemargin}{0cm}
\topmargin -1.25cm
\footskip 1.4cm

\linespread{1.3}

\begin{document}
	\raggedbottom 
	%\input{hypernation}
	%\input{title-page}
	\setlength\extrarowheight{2pt}
	\begin{table}[ht]
		\centering
		%\resizebox{\textwidth}{!}{%
			\begin{tabular}{|p{5cm}|p{7cm}|p{2cm}|}
				\hline
				%\rowcolor{gray}
				%\multicolumn{3}{|c|}{\textcolor[rgb]{1,1,1}{Scalenie trzech kolumn}}\\
				
				\multicolumn{2}{|c|}{\cellcolor{gray-pp}\textcolor[rgb]{1,1,1}{Politechnika Poznańska}}  & \multicolumn{1}{c|}{\multirow{3}{*}{\resizebox{15mm}{!}{\includegraphics{PP_znak_konturowy_CMYK.pdf}}}}\\ 
				\multicolumn{2}{|c|}{\cellcolor{gray-pp}\textcolor[rgb]{1,1,1}{Wydział Automatyki, Robotyki i Elektrotechniki}} & \\ 
				\multicolumn{2}{|c|}{\cellcolor{gray-pp}\textcolor[rgb]{1,1,1}{Instytut Robotyki i Inteligencji Maszynowej}} & \\ 
				\hline 
				\multicolumn{1}{|c|}{Dz>AiR>Sem5} & \multicolumn{1}{c|}{Napędy przekształtnikowe (NP)} & \multicolumn{1}{c|}{2025/26 (s.zim.)} \\
				\hline
				\textbf{Skład osobowy:} \par Zuzanna Andrzejak 159522 \par Jan Andrzejewski 159512 \par Mateusz Banaszak 159416 \par Piotr Bednarek 159701 & 
				\textbf{Identyfikacja parametrów modelu obwodowego silnika prądu stałego} 
				& Data wyk.:\par 04.12.2025\\
				\hline
				Grupa 1  & Ćwiczenie 2 & Zajęcia 2 \\
				\hline
			\end{tabular}%
			%}
	\end{table}	
	\setlength\extrarowheight{0pt}
	\vspace{1.5cm}
	\tableofcontents
	\newpage
	
	\section{Wprowadzenie}
	W poprzednim ćwiczeniu laboratoryjnym zdefiniowano model obwodowy obcowzbudnego silnika prądu stałego z komutatorem, przyjmując określone parametry w celu zrealizowania uproszczonego modelu. Celem kolejnego ćwiczenia było eksperymentalne wyznaczenie tych parametrów na podstawie pomiarów przeprowadzonych na rzeczywistym, identyfikowanym obiekcie \cite{ekurs}. Podczas analizy uwzględniono wektor parametrów modelu w postaci:
	\begin{equation}
		P^T = \left[ R_a,\; L_a,\; k_{\Phi},\; J_r,\; b_1 \right]
		\label{eq:parametry}
	\end{equation}
	
	
	Pomiary zrealizowano w sali C3 w budynku A22b, wyposażonej w stanowisko laboratoryjne z układem silnika umożliwiającym wymuszenie zadanej prędkości obrotowej wału. Stanowisko pomiarowe obejmowało ponadto multimetry, laboratoryjny zasilacz z ograniczeniem prądu oraz oscyloskop. W trakcie ćwiczenia wykorzystano również środowisko symulacyjne \texttt{MATLAB} do analizy zarejestrowanych danych pomiarowych oraz sporządzenia odpowiednich charakterystyk i wykresów.

	\vspace{1cm}
	
	\section{Rezystancja twornika $R_a$}
	\subsection{Metodyka pomiarów}
	W celu wyznaczenia rezystancji uzwojenia twornika $R_a$ zastosowano \textbf{metodę techniczną}, cechującą się mniejszą niepewnością pomiarową w porównaniu z pozostałymi metodami. W układzie pomiarowym wykorzystano laboratoryjny zasilacz \texttt{MCP M10-TP-303E}, umożliwiający zadawanie określonych wartości prądu zasilania. Wartości prądu odczytywano bezpośrednio z wyświetlacza zasilacza.
	
	Ze względu na niepewność pomiarową związaną z odczytem napięcia z wyświetlacza zasilacza, pomiar napięcia realizowano za pomocą multimetru \texttt{UNI-T UT58C}.
	
	Rezystancję przewodów pomiarowych pominięto, ponieważ zastosowano przewody wysokiej klasy, a ich długość była niewielka ze względu na bezpośrednie umiejscowienie aparatury pomiarowej w pobliżu badanego obwodu, co skutkowało znikomym wpływem na końcowy wynik pomiaru.
	
	\subsection{Pomiary i opracowanie wyników}
	
	Zebrano pomiary prądu uzwojenia twornika $I_a$ oraz napięcia twornika $U_a$. W trakcie badań zmieniano wartość prądu zasilania w zakresie do $3\ A$. Na podstawie uzyskanych danych pomiarowych możliwe było wyznaczenie rezystancji uzwojenia twornika dwiema niezależnymi metodami.
	\\
	
	Pierwsza metoda polegała na obliczeniu wartości rezystancji dla każdego punktu pomiarowego na podstawie prawa Ohma, zgodnie z zależnością:
	\begin{equation}
		R_a = \frac{U_a}{I_a}
		\label{eq:ohm}
	\end{equation}
	gdzie: $R_a$ to rezystancja twornika, $U_a$ to napięcie na zaciskach twornika, $I_a$ to prąd uzwojenia twornika.
	
	Otrzymane wyniki zestawiono w tabeli \ref{tab:rezystancja}.
	\vspace{1cm}
		
	\begin{table}[h]
		\centering
		\caption{Wyniki pomiarów i obliczona rezystancja}
		\label{tab:rezystancja}
		\begin{tabular}{|c|c|c|}
			\hline
			Prąd $I_a$ [A] & Napięcie $U_a$ [V] & Rezystancja $R_a$ [$\Omega$] \\
			\hline
			0.00 & 0.00 & 0.00 \\
			0.50 & 1.53 & 3.06 \\
			0.74 & 2.24 & 3.03 \\
			1.00 & 3.04 & 3.04 \\
			1.25 & 3.78 & 3.02 \\
			1.50 & 4.65 & 3.10 \\
			1.75 & 5.28 & 3.02 \\
			2.00 & 6.19 & 3.10 \\
			2.25 & 6.82 & 3.03 \\
			2.50 & 7.71 & 3.08 \\
			2.75 & 8.42 & 3.06 \\
			2.99 & 9.18 & 3.07 \\
			
			\hline
		\end{tabular}
	\end{table}
	
	Następnie wyznaczono wartość średnią arytmetyczną rezystancji na podstawie wszystkich przeprowadzonych pomiarów:
	\begin{equation}
		R_a \approx 3.0555 \ \Omega
		\label{eq:Ra_srednia}
	\end{equation}
	co stanowiło pierwsze przybliżenie wartości parametru $R_a$.
	\\
	
	\begin{figure}[h!]
		\centering
		\includegraphics[width=0.8\textwidth]{ra.pdf}
		\caption{Wyznaczenie rezystancji twornika metodą regresji liniowej}
		\label{fig:Ra}
	\end{figure}
	
	Drugą metodą wyznaczenia rezystancji uzwojenia twornika było zastosowanie regresji liniowej. Linię regresji wyznaczono z wykorzystaniem skryptu w środowisku \texttt{MATLAB}. Współczynnik kierunkowy dopasowanej charakterystyki stanowił poszukiwaną wartość rezystancji. Regresję przeprowadzono w postaci funkcji liniowej $y = ax$, bez uwzględnienia wyrazu wolnego, co pozwoliło ograniczyć wpływ niepożądanych zakłóceń i błędów offsetu pomiarowego. Charakterystyka została przedstawiona na rysunku \ref{fig:Ra}
	
	Wartość rezystancji twornika wyznaczona metodą regresji liniowej wyniosła:
	\begin{equation}
		R_a = 3.0724 \ \Omega
		\label{eq:Ra_regresja}
	\end{equation}
	
	Uzyskane wartości rezystancji różnią się ze względu na odmienny charakter błędów pomiarowych występujących w obu metodach. Metoda regresji liniowej charakteryzuje się mniejszą wrażliwością na błędy losowe, ponieważ wykorzystuje globalną informację zawartą w całym zbiorze punktów pomiarowych. Analizowana charakterystyka opisuje wzajemnie powiązane punkty, których trend oceniany jest w całym zakresie pomiarowym, w przeciwieństwie do metody średniej arytmetycznej, w której każdy pomiar traktowany jest niezależnie.
	
	W związku z powyższym, jako bardziej wiarygodną metodę wyznaczania rezystancji uzwojenia twornika przyjęto metodę regresji liniowej, która zapewnia mniejszą niepewność pomiarową oraz lepsze odwzorowanie rzeczywistej zależności pomiędzy mierzonymi wielkościami.
	
	\subsection{Porównanie otrzymanych wartości}
	
	Dla ilościowego porównania obu metod wyznaczono różnicę bezwzględną oraz względną pomiędzy uzyskanymi wartościami rezystancji uzwojenia twornika. Różnica bezwzględna wynosi
	\begin{equation}
		\Delta R_a = \left| R_{a \ reg} - R_{a \ śr} \right| = 0.0169 \ \Omega
	\end{equation}
	
	Różnicę względną, odniesioną do wartości uzyskanej metodą regresji liniowej, wyznaczono ze wzoru
	\begin{equation}
		\delta R_a = \frac{\Delta R_a}{R_{a \ reg}} \cdot 100\% \approx 0.55\%
	\end{equation}
	
	Niewielka wartość różnicy względnej potwierdza spójność otrzymanych wyników oraz poprawność przeprowadzonych pomiarów. Jednocześnie metoda regresji liniowej zapewnia większą odporność na błędy losowe i lepsze wykorzystanie informacji zawartej w całym zbiorze danych pomiarowych, dlatego jej wynik został przyjęty do dalszych analiz.
	
	\vspace{1cm}
	
	\section{Indukcyjność uzwojenia twornika $L_a$}
	\subsection{Metodyka pomiarów}
	
	Pomiar indukcyjności uzwojenia twornika $L_a$, w przeciwieństwie do pomiaru rezystancji, możliwy jest wyłącznie w stanie dynamicznym. Efekt indukcyjności ujawnia się w odpowiedzi układu na wymuszenie nieustalone, w szczególności o charakterze skokowym. Ze względu na właściwości zastosowanego zasilacza laboratoryjnego uzyskanie idealnego wymuszenia skokowego było utrudnione, ponieważ obecność transformatora powodowała istotne zakłócenia w momentach przełączania przekaźników.
	
	Z tego względu do obserwacji wpływu indukcyjności uzwojenia twornika wykorzystano \textbf{zjawisko zaniku prądu w obwodzie}, występujące w procesie komutacji. Analiza przebiegu czasowego prądu w fazie jego wygaszania umożliwiła pośrednie wyznaczenie wartości indukcyjności $L_a$.
	\\ 
	
	Proces zaniku prądu w obwodzie twornika ma charakter wykładniczy i zachodzi od wartości początkowej $A$ do zera ze stałą czasową $\tau_a$. Przebieg ten można aproksymować przy pomocy równania:
	\begin{equation}
		I_a(t) = A e^{-\frac{t}{\tau_a}}
		\label{eq:Ae}
	\end{equation}
	
	Dla każdego układu typu $RL$ stała czasowa $\tau_a$ określona jest zależnością:
	\begin{equation}
		\tau_a = \frac{L_a}{R_a}
		\label{eq:RL}
	\end{equation}
	
	Oznacza to, że wyznaczając chwilę czasu, w której wartość prądu osiąga poziom $\frac{A}{e}$, możliwe jest określenie stałej czasowej układu, gdyż zachodzi zależność:
	\begin{equation}
		I_a(t = \tau_a) = A e^{-1} = \frac{A}{e}
	\end{equation}
	
	Po wyznaczeniu wartości stałej czasowej $\tau_a$ oraz znanej rezystancji uzwojenia twornika $R_a$, określonej w poprzednim punkcie, indukcyjność uzwojenia twornika jest możliwa do obliczenia z przekształconej zależności \eqref{eq:RL}:
	\begin{equation}
		L_a = \tau_a R_a
		\label{eq:La}
	\end{equation}
	
	\subsection{Pomiary i opracowanie wyników}
	
	W celu wyznaczenia indukcyjności uzwojenia twornika $L_a$ utworzono układ pomiarowy z wykorzystaniem wcześniej stosowanego zasilacza laboratoryjnego, umożliwiającego wymuszenie zadanej wartości prądu twornika. Układ pomiarowy składał się także z badanego obwodu typu $RL$ oraz przełącznika sterującego.
	\\
	
	Ponieważ celem pomiaru była obserwacja odpowiedzi dynamicznej układu, konieczne było wytworzenie wymuszenia skokowego. Zrealizowano je poprzez gwałtowne zwarcie obwodu za pomocą przełącznika, co powodowało zanik prądu w obwodzie pomiarowym. Przebieg wygaszania prądu stanowił podstawę do dalszej analizy indukcyjności uzwojenia twornika.
	
	Pomiary przeprowadzono dla trzech nastaw wartości prądu początkowego: 
	$A = I_a = 1.01\,\mathrm{A}$, $2.02\,\mathrm{A}$ oraz $3.10\,\mathrm{A}$. 
	Dane niezbędne do wyznaczenia indukcyjności uzwojenia twornika zarejestrowano za pomocą oscyloskopu w postaci plików \texttt{.csv}.
	\\
	
	\begin{figure}[h!]
		\centering
		\includegraphics[width=0.8\textwidth]{1a.pdf}
		\caption{Przebieg zaniku prądu w obwodzie twornika, $I_a = 1.01$ A}
		\label{fig:La_1a}
	\end{figure}
	
	\begin{figure}[h!]
		\centering
		\includegraphics[width=0.8\textwidth]{2a.pdf}
		\caption{Przebieg zaniku prądu w obwodzie twornika, $I_a = 2.02$ A}
		\label{fig:La_2a}
	\end{figure}
	
	\begin{figure}[h!]
		\centering
		\includegraphics[width=0.8\textwidth]{3a.pdf}
		\caption{Przebieg zaniku prądu w obwodzie twornika, $I_a = 3.10$ A}
		\label{fig:La_3a}
	\end{figure}
	
	Na podstawie zarejestrowanych przebiegów, z wykorzystaniem dedykowanego skryptu w środowisku \texttt{MATLAB}, opracowano charakterystyki czasowe prądu, przedstawione na rysunkach \ref{fig:La_1a}, \ref{fig:La_2a}, \ref{fig:La_3a}, na podstawie których wyznaczono wartości stałej czasowej układu:
	\begin{equation}
		\label{eq:tau}
		\tau_a =\begin{bmatrix}
			23.0 \\
			21.3 \\
			20.0
		\end{bmatrix} 
		\ \mathrm{ms}
	\end{equation}

	Na ich podstawie, zgodnie z zależnością \eqref{eq:La}, obliczono wartości indukcyjności uzwojenia twornika:
	\begin{equation}
		L_a =
		\begin{bmatrix}
			70.665 \\
			65.442 \\
			61.448
		\end{bmatrix}
		\ \mathrm{mH}
	\end{equation}
	
	Następnie wyznaczono średnią arytmetyczną otrzymanych wartości indukcyjności:
	\begin{equation}
		\bar{L}_a = 65.852 \ \mathrm{mH}
	\end{equation}
	
	W celu oceny rozrzutu wyników obliczono odchylenie standardowe:
	\begin{equation}
		s = \sqrt{\frac{1}{N-1}\sum_{i=1}^{N}\left(L_{a \ i}-\bar{L}_a\right)^2}
		= 4.622 \ \mathrm{mH}
	\end{equation}
	gdzie $N=3$ oznacza liczbę przeprowadzonych pomiarów.
	
	
	Ostatecznie wartość \textbf{indukcyjności uzwojenia twornika} określono jako:
	\begin{equation}
		L_a = (65.852 \pm 4.622)\ \mathrm{mH}
	\end{equation}
	
	
	\vspace{1cm}
	
	
	\section{Współczynnik $k_{\Phi}$}
	\subsection{Metodyka pomiarów}
	
	Z poprzedniego ćwiczenia laboratoryjnego wiadomo, że współczynnik $k_{\Phi}$ stanowi uproszczenie przyjęte w analizowanym modelu silnika. W ujęciu rzeczywistym rozróżnia się stałą napięciową $k_v$ oraz stałą momentową $k_m$. W niniejszym ćwiczeniu nadal rozpatrywana jest uproszczona postać parametru, zakładająca równość obu stałych.
	\\
	
	Przy takim założeniu współczynnik $k_{\Phi}$ może zostać wyznaczony na podstawie dwóch zależności opisujących pracę silnika prądu stałego:
	\begin{equation}
		T_e = k_{\Phi} \cdot i_a
		\label{eq:kfi_moment}
	\end{equation}
	\begin{equation}
		\varepsilon = k_{\Phi} \cdot \omega_r
		\label{eq:kfi_napiecie}
	\end{equation}
	gdzie $T_e$ oznacza moment elektromagnetyczny, $i_a$ to prąd twornika, $\varepsilon$ to siła elektromotoryczna, $\omega_r$ to prędkość kątową wirnika.
	\\
	
	Wybór metody wyznaczenia współczynnika $k_{\Phi}$ podyktowany był możliwościami oferowanymi przez stanowisko laboratoryjne. Bezpośredni pomiar momentu elektromagnetycznego jest procesem złożonym i obarczonym dużą niepewnością, natomiast pomiar siły elektromotorycznej oraz prędkości obrotowej jest znacznie prostszy do realizacji. Z tego względu do wyznaczenia współczynnika $k_{\Phi}$ wykorzystano zależność \eqref{eq:kfi_napiecie}, którą przekształcono do postaci:
	\begin{equation}
		\label{eq:kfi_wyznaczenie}
		k_{\Phi} = \frac{\varepsilon}{\omega_r}
	\end{equation}
	
	Takie wyznaczenie współczynnika $k_{\Phi}$ jest możliwe wyłącznie w przypadku, gdy napięcie na zaciskach twornika spełnia zależność:
	\begin{equation}
		U_a = \varepsilon
	\end{equation}
	Warunek ten zachodzi jedynie przy rozwartym obwodzie twornika, gdy prąd twornika jest równy zeru, a spadki napięcia na rezystancji i indukcyjności uzwojenia nie występują.
	
	Ponieważ podczas pomiaru zaciski twornika muszą pozostać rozwarte, a jednocześnie konieczne jest zapewnienie niezerowej prędkości obrotowej wirnika, realizacja pomiarów wymagała zastosowania aktywnego wzbudzenia mechanicznego. W tym celu do napędu obcego wykorzystano urządzenie \texttt{Microverter AEG}, pracujące jako przekształtnik częstotliwości, umożliwiający sterowanie prędkością obrotową silnika napędzającego badany obiekt.
	
	\subsection{Pomiary i opracowanie wyników}
	
	W celu wyznaczenia stałej $k_{\Phi}$ z zależności \eqref{eq:kfi_wyznaczenie} przeprowadzono pomiary zależności siły elektromotorycznej $\varepsilon$ od zadanej prędkości obrotowej $\omega_r$. Prędkość obrotowa była wymuszana za pomocą przekształtnika częstotliwości \texttt{Microverter AEG}, pracującego jako napęd obcy.
	
	Ponieważ analizowana charakterystyka ma charakter liniowy, do wyznaczenia współczynnika $k_{\Phi}$ zastosowano metodę regresji liniowej. W przeciwieństwie do wcześniejszych analiz, wykorzystano model regresji liniowej z wyrazem wolnym. Decyzję tę podjęto ze względu na występowanie stałego przesunięcia (offsetu) toru pomiarowego napięcia twornika, który w badanym układzie powodował pojawienie się napięcia rzędu $0.02\,\mathrm{V}$ przy zerowej prędkości obrotowej.
	
	\begin{figure}[h!]
		\centering
		\includegraphics[width=0.8\textwidth]{kfi.pdf}
		\caption{Wyznaczenie stałej $k_{\Phi}$ metodą regresji liniowej}
		\label{fig:kfi}
	\end{figure}
	
	Uwzględniając wyraz wolny w modelu regresji liniowej, otrzymano równanie opisujące zależność siły elektromotorycznej od prędkości obrotowej:
	\begin{equation}
		\varepsilon = 1.800 \ \omega_r + 0.02 \ \mathrm{[V]}
	\end{equation}
	
	Współczynnik kierunkowy otrzymanej charakterystyki odpowiada wartości współczynnika $k_{\Phi}$, który dla analizowanego silnika wynosi:
	\begin{equation}
		k_{\Phi} = 1.800
	\end{equation}
	
	
	
	
	
	
	\vspace{1cm}
	\section{Moment bezwładności wirnika $J_r$}
	\vspace{1.5cm}
	
	\section{Współczynnik $b_1$}
	

	
	
	\vspace{1.5cm}
	
	\begin{thebibliography}{9}
		\bibitem{ekurs}
		Materiały do ćwiczenia laboratoryjnego dostępne na platformie eKursy, Politechnika Poznańska, \texttt{https://ekursy.put.poznan.pl/mod/folder/view.php?id=3022726}
		\bibitem{zawirski}
		K. Zawirski, J. Deskur, T. Kaczmarek, \textit{Automatyka napędu elektrycznego}, Wydawnictwo Politechniki Poznańskiej, 2012
		\bibitem{sidorowicz}
		J. Sidorowicz, \textit{Napęd elektryczny i jego sterowanie}, Oficyna Wydawicza Politechniki Warszawskiej, 1997
	\end{thebibliography}
	
	
	
	
\end{document}